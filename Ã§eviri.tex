% vim: set filetype=tex fileencoding=utf8 et sw=2 ts=2 sts=2 tw=80 :
% © 2014 Ali Polatel <polatel@gmail.com>
% Creative Commons Attribution-NonCommercial-ShareAlike 3.0 Unported Lisansı ile yayınlanmıştır.

\chapter{Çeviri: ``Siz Zombiler''}

\noindent\emph{22.17-5. Zaman Dilimi-7 Kasım 1970-New York, ``Baba'nın Yeri'':}

Evlenmemiş Anne içeri girdiğinde bir kanyak kadehini temizliyordum. Zamanı not
ettim: 7 Kasım 1970, beşinci -- veya batı -- zaman diliminde saat akşam 10.17.
Zaman Ajanları tarih ve zamana hep dikkat eder; mecburiyetten.

Evlenmemiş Anne 25 yaşındaydı. Boyu benden uzun değildi. Çocukça yanları olan
alıngan bir adamdı. Tipi hoşuma gitmedi, hiç de gitmemişti zaten. Ama onu işe
almak için buradaydım. Adamımdı. Onu sıcak bir barmen gülümsemesiyle karşıladım.

Fazla abartıyorum belki. Yumuşak falan değildi. Ne zaman meraklının biri gelip
işini gücünü sorsa ``Ben evlenmemiş bir anneyim.'' derdi. Adamı öldüresi
gelmezse de ``kelimesi dört kuruşa itiraf mektupları yazarım.'' diye eklerdi.
Adı oradan geliyordu.

Heyheyleri üstündeyken çatacak adam arardı. Yakın mesafeden ölümcül dövüşürdü,
kadın polisler gibi. Onu istememin bir sebebi de buydu, tek sebebi değil ama.

Geldiğinde dut gibiydi. Etrafına ayrı bir nefretle baktığı belli oluyordu.
Önüne sessizce bir duble Don Murdar koydum. Şişeyi de yanına bıraktım. Dikip
kafaya yeniden doldurdu.

Tezgâhı sildim. ``Şu `Evlenmemiş Anne' dümeni nasıl gidiyor?'' diye sordum.

Elindeki kadehi bana atacakmış gibi sıktı. Ben de tezgâhın altındaki sopayı
yokladım. Zamanı kontrol ediyorsan her ihtimali bilirsin ama hesaba katacak o
kadar çok değişken vardır ki asla gereksiz yere risk almazsın.

Büro'nun eğitim okulunda dikkat etmeni söyledikleri şu küçük oran kadar
yatıştığını gördüm. ``Kusura bakma,'' dedim. ```İşler nasıl?' diye sordum
sadece. `Havalar nasıl?' diye sormuşum farz et.''

Suratını ekşitti. ``İdare eder. Ben yazıyorum, onlar basıyor. Aç kalmıyorum.''

Kendime bir kadeh doldurup yanına yanaştım. ``Aslına bakarsan,'' dedim.
``yazdıkların hiç fena değil. Birkaçının kopyasını aldım. Kadın bakış açısından
şaşırtıcı derecede iyi anlıyorsun.''

Bunu söylerken riski göze almıştım. Kullandığı mahlasları kimseye söylemezdi.
Ama öyle sarhoştu ki ancak sonunu yakalayabilmişti. ``Kadın bakış açısı ha!''
diye tekrar etti homurdanarak. ``Ben bilmeyeceğim de kim bilecek!''

``Yani,'' dedim tereddüt ederek. ``Kız kardeşin mi var?''

``Hayır. Söylesem de inanmazsın zaten.''

``Hadi ama,'' diye cevapladım yumuşak bir sesle. ``Meyhanecilerin ve
psikiyatristlerin bildiği ortak bir şey varsa o da hiçbir şeyin gerçek kadar
garip olmadığıdır. Neden biliyor musun evlat, benim duyduğum hikayeleri kağıda
döksen zengin olursun. Akıl alır şeyler değil.''

```Akıl almaz' ne demek bilmiyorsun sen!''

``Anlat madem. Beni şaşırtamazsın. Kesin daha kötüsünü duymuşumdur.''

Yine burnundan solumaya başladı. ``Şişenin geri kalanına bahse var mısın?''

``Geri kalanını bırak, yeni bir şişesine girerim iddiaya.'' dedikten sonra
açılmamış bir şişe koydum tezgâha ve ``Anlat bakalım.'' dedim.

Garsona işlere göz kulak olması için bir işaret çaktım. Barın bir ucunda turşu
kavanozlarıyla diğer ıvır zıvırları dizip ayırdığım özel köşeme geçtik. Öbür
tarafta birkaç kişi at yarışı izliyor, biri de müzik kutusuna para atıyordu.
Burada rahat rahat konuşabilirdik.

``Pekala.'' diyerek söze başladı. ``Öncelikle ben bir piçim.''

``Burada öyle bir ayrım yaptığımız yok.'' dedim.

``Ben ciddiyim.'' diye çıkıştı. ``Annemle babam evli değildi.''

``Ne olmuş yani?'' diye üsteledim. ``Bizimkiler de evli değildi.''

``Ne,--'' derken duraksadı. Gözlerinde ilk defa dostça bir bakış görüyordum.
``Gerçekten mi?'' diye sordu.

``Tabi ya.'' diye yanıtladım. ``Yüzde yüz bir piçim hem de. Aslına bakarsan
bizim ailede kimse evlenmez. Hepimiz piçiz.''

Gözü elimdeki yüzüğe takıldı. ``Buna mı bakıyorsun?'' diyerek ona gösterdim.

``Evlilik yüzüğüne benzemesi seni aldatmasın. Kadınları uzak tutmak için
takıyorum.'' dedim.  Bu antik yüzüğü benim gibi ajan olan bir meslektaşımdan
1985 yılında satın almıştım. O da milattan önce bir zaman Kıbrıs'tan almış.
``Ouroboros Solucanı\dots Sonu olmayan, sonsuza dek kendi kuyruğunu yiyen koca
yılan. Büyük Paradoks'un simgesiymiş.''

Pek ilgisini çekmedi. ``Eğer gerçekten bir piçsen nasıl bir his olduğunu
bilirsin. Ben küçük bir kızken--''

``Doğru mu duydum? Kimin ağzından anlatıyorsun bu hikayeyi? Demek sen küçük bir
kızken\dots Christine Jorgensen adını duydun mu hiç? Ya da Roberta Cowell'ı?
Bu muydu yani? Cinsiyet mi değiştirdin?''

``Bak bir daha sözümü kesersen anlatmam. 1945 senesinde henüz bir aylıkken
Cleveland'da bir yetimhaneye bırakılmışım. Küçüklüğüm ana babası olan çocuklara
imrenerek geçti. Sonra cinsellik konusunu öğrendiğimde\dots ve inan bana babalık
yetimhanede böyle şeyleri çabuk öğrenirsin.''

``Bilirim.''

``İşte o zaman çocuklarımın hiçbirini ana babasız bırakmayacağıma dair yemin
ettim. Böylece orada `saf' kalabildim. Öyle kolay bir iş değil o ortamda.
Mücadele etmeyi öğrenmem gerekti. Büyüyünce fark ettim ki evlatlık alınma şansım
olmadığı gibi evlenme şansım da neredeyse hiç yoktu\dots İkisi de aynı
sebepten.''

Kaşlarını çattı. ``Domuz yüzlü, dişlek, kütük memeli ve düz saçlı bir tiptim.''

``Benden kötü görünmüyorsun.''

``Bir meyhanecinin nasıl göründüğünü kim takar ki? Ya da bir yazarın? Ama evlat
edinmek isteyenler altın saçlı, mavi gözlü moronları seçer. Daha sonra da koca
memeli, tatlı suratlı ve hani şu `muhteşem erkek tavırlı' kızlar peşine takar
herkesi.''

Omuz silkti. ``Böylelerinin yanında esamem okunmazdı. Ben de K.A.L.T.A.K'a
katılmaya karar verdim.''

``Ha?''

``Kadın Astronot Levazım Teşkilatı Ağırlama Kurumu, bugünlerde adı `Uzay
Melekleri' diye geçiyor: Uzay Mangaları için Eğlence, Levazım ve Eğitim
Komutanlığı.''

Bahsettiği tabirlere aşinaydım. Vaktiyle ikisinin de zaman kayıtlarını
çıkarmıştım. Bu seçkin askeri hizmet teşkilatı için bizim kullanıdığımız üçüncü
bir isim de vardı: Kadın Astronot Nizamiyesi Cezbe-i İtidal Kurulu.

Zaman sıçramalarının en büyük zorluklarından biri de kelime anlamlarının
değişmesidir. ``Servis istasyonu''nun bir zamanlar akaryakıt satılan yer
anlamına geldiğini biliyor muydunuz? Bir keresinde Churcill döneminde
görevdeyken kadının biri bana ``Bitişikteki servis istasyonunda buluşalım.''
demişti; kulağa nasıl geldiğini biliyorum ama (o zamanlar) ``servis
istasyonu'', içinde yatak olacak bir yer değildi.

Anlatmaya devam etti: ``Aylar, yıllar boyunca uzayda görev yapacak bu erkekleri
rahatlatmaları gerektiğini anladıkları zamanlardı. Sofuların ne yaygara
kopardığını hatırlamıyor musun? Bu tantana bana yaramıştı; başvuranların sayısı
yok denecek kadar azdı. Eli yüzü düzgün, tercihen bâkire (eğitime sıfırdan
başlamayı seviyorlardı), zekâ olarak ortalamanın üstünde ve sinirlerine hakim
olabilen kızlar arıyorlardı ama başvuranların çoğu ya eski fahişeydi ya da
dünyadan on gün uzak kalsa kafayı sıyıracak dengesiz tiplerdi. Anlayacağın işi
bana vermeleri için güzel görünmeme gerek yoktu: Tavşan dişlerimi düzeltip
saçlarımı yapacaklardı. Bana düzgün yürümeyi, dans etmeyi ve erkekleri nasıl
memnun edeceğimi öğreteceklerdi. Üstüne temel görevlerim için eğitim de
alacaktım. Hatta gerekirse estetik ameliyat bile olacaktım. Evlatlarımıza
Canımız Feda.''

``En güzeli de verdiğin hizmet boyunca hamile kalmamanı sağlıyorlardı ve görevin
bitince de evlenmen garanti gibiydi. Tıpkı bugün olduğu gibi Uzay Melekleri,
astronotlarla evlenirdi. Aynı dili konuştukları kesin.''

``On sekiz yaşındayken bana bir dadılık işi buldular. Aslında yanında çalıştığım
ailenin istediği sıradan bir hizmetçiden başka bir şey değildi ama yirmi bir
yaşından önce askere yazılamıyordum. Gündüzleri ev işleriyle ilgileniyor sonra
da gece okuluna gidiyordum. Evdekilere daktilo ve stenografi derslerine devam
ediyorum desem de aslında Melekler'e kabul edilme şansımı arttırmak için zerafet
dersleri alıyordum.''

``Sonra bir gün şehirli bir çocukla tanıştım. Cebi yüz dolarlık banknotlarla
doluydu.'' dedi ve kaşlarını çattı. ``Serserinin gerçekten yığınla parası vardı.
Bir keresinde cebinden bir tomar çıkarıp bana vermeye kalktı.''

``Kabul etmedim. Çocuğa abayı yakmıştım. İlk defa bir erkek benimle alay
etmiyor, bana iyi davranıyordu. Onu daha sık görebilmek için gece okulunu
bıraktım. Hayatımın en güzel günleriydi.''

``Sonra bir gece parkta mercimeği fırına verdik.'' dedi ve duraksadı.

``Ya sonra?'' diye sordum.

``Sonrası \emph{yok!} Onu bir daha görmedim. Beni eve bıraktı, sevdiğini söyledi, öptü
ve gitti. Gidiş o gidiş.''

Çok sinirlenmişti. ``Onu bulursam öldüreceğim!'' dedi.

``Elbette,'' dedim duygularına ortak olarak ve devam ettim: ``nasıl hissettiğini
anlıyorum ama yalnızca içinden geldiği gibi davrandığı için onu öldürmek\dots Çok
içerledin herhalde?''

``Ha? Ne alakası var şimdi?''

``Var tabi, olmaz olur mu. Belki seni öyle bıraktığı için iki tokadı
hak ediyordur ama\dots''

``Sandığından çok daha fazlasını hak ediyor! Bir de şunu dinle. Bu olayı
herkesten sakladım ve olacağına vardı deyip geçtim. Ona gerçekten aşık değildim
ve muhtemelen bir daha aşık da olmayacaktım. Artık K.A.L.T.A.K'a katılmayı her
şeyden çok istiyordum. Şansım yaver gitti: bâkire olmadığımdan dolayı
elemediler. Yine yüzüm gülüyordu.''

``Başımın belada olduğunu eteklerim dar gelmeye başlayınca farkettim.''

``Hamile mi kaldın?''

``Kandırdığı yetmemiş gibi bir de hamile bırakmış beni! Pinti ev sahiplerim
çalışabildiğim sürece buna aldırış etmediler ama sonra beni kapının önüne
koydular.  Yetimhane de beni geri almıyordu. Sonunda bir sığınma evine düştüm.
Bir sürü koca göbekli kadın, kıçımızda lazımlıklarla günümüzün gelmesini
bekliyorduk.''

``Bir gece kendimi ameliyat masasında buldum. Yanı başımdaki hemşire bana
`Rahatla. Şimdi derin nefes al.' diyordu.''

``Yatağımda uyandığımda göğsümden aşağısı uyuşmuş haldeydi. Doktorum odaya
geldi. Yüzünde gülen bir ifadeyle `Kendini nasıl hissediyorsun?' diye sordu.''

```Bir anne gibi.'

```Doğal tabi. Sargılar içindesin ve bir şey hissetmemen için sana epey narkoz
verdik. İyileşmen biraz zaman alabilir. Sezaryen diş çektirmeye benzemez.'

```Sezaryen.' diye yineledim. `Doktor, yoksa bebeğime bir şey mi oldu?'

```Yo, hayır. Bebeğin iyi durumda.'

```Erkek mi kız mı peki?'

```Sağlıklı küçük bir kızın oldu. İki kilo, üç yüz elli gram.'

``İçim rahatlamıştı. Bir bebeğim olmuştu. `Bu da bir şey' diye geçirdim içimden.
Kimliğine dul diye yazdırıp çocuğa da babasının öldüğünü söylerim dedim.
Bebeğimi bir yetimhaneye terk edemezdim!''

``Fakat doktorun söyleyecekleri bitmemişti. `Söyler misin,\dots' Adımı
söylemekten kaçınıyordu. `Tuhaf bir hormonal yapın olduğunu fark etmiş miydin?'

```Ha? Tabi ki hayır. Nereye varmaya çalışıyorsun?' dedim. Çekinerek `Bunu sana bir
seferde söyleyeceğim. Sonra da yatışman için seni uyutacak bir şeyler vereceğim.
Merak etme.' dedi.''

```Neden?' diye ısrar ettim.''

```Şu otuz beş yaşına kadar cinsiyeti kadın olan İskoç doktoru duymuş muydun?
Sonrasında ameliyat olup hem fiziksel hem de yasal olarak bir erkek olmuştu.
Evlendi de üstelik bir sorun yaşamadan.'

```Bunun benimle ne alakası var?'

```Söylemeye çalıştığım bu. Sen artık bir erkeksin.'

``Doğrulmaya çalıştım. \emph{`Neyim?'}

```Sakin ol. Ameliyat için vücudunu açtığımda içerisinin karmakarışık olduğunu
gördüm. Bebeği çıkarırken baş cerraha da haber yolladım. Sen masada baygın
yatarken konsültasyon yaptık. Ardından da saatlerce kurtarabileceğimiz kadarını
kurtarmak için çabaladık. Hem dişi hem de erkek cinsiyet organlarına sahiptin.
Henüz olgunlaşmamışlardı fakat dişi cinsiyet organların bir bebek sahibi olmana
yetecek kadar gelişmişlerdi. Bir daha işine yaramayacakları için bu organları
çıkardık ve erkek organlarını da sağlıklı bir erkek olarak gelişebileğin şekilde
düzenledik.' Elini omzuma koydu. `Endişelenme. Gençsin, kemiklerin uyum
sağlayacaktır. Hormonal dengeni de takip edeceğiz. Sağlıklı bir delikanlı
olacaksın.'

``Ağlamaya başladım. `Peki ya \emph{bebeğim} ne olacak?'

```Onu emziremezsin. Bir kedi yavrusuna yetecek kadar bile sütün yok. Senin
yerinde olsam onu bir daha görmezdim. Evlatlık verirdim.'

``\emph{`Asla!'}''

``Omuz silkti. `Karar senin. Çocuğun annesi, yani ebeveyni, sensin. Şimdilik
bunları düşünüp kafanı yorma. Önce seni iyi edelim.'

``Ertesi gün bana bebeğimi gösterdiler. Onu her gün görüyor ve alışmaya
çalışıyordum. Daha önce hiç yeni doğmuş bir bebek görmemiştim ve bu kadar çirkin
oldukları aklımın ucundan geçmezdi. Kızım turuncu bir maymuna benziyordu.
Duygularımın yerini soğuk bir kararlılık almıştı. Onun için doğru olanı
yapacaktım. Gerçi dört hafta sonra artık bunun da hiçbir anlamı kalmadı.''

``Ha?''

``Onu götürdüler.''

``Götürdüler mi?''

Evlenmemiş Anne, üzerine iddiaya girdiğimiz şişeyi az kalsın düşürüyordu.

``Kaçırıldı. Hastanenin bakım odasından çalındı!'' Nefes almakta zorlanıyordu.
``Bir adamın hayatına anlam veren son şeyi nasıl alırsın ya?''

``Gerçekten fena.'' diyerek söylediklerine hak verdim. ``Buyur bir kadeh daha
iç. Bebekten haber aldınız mı?''

``Polisin takip edebileceği bir ipucu yoktu. Biri onu görmeye gelmiş, dayısı
olduğunu söylemiş. Hemşirenin arkası dönükken de bebeği alıp gitmiş.''

``Eşkâli?''

``Öyle sıradan bir adam, sen ben gibi.'' Kaşlarını çattı. ``Bence bebeğin
babasıydı. Hemşire yaşlı bir adam olduğuna yemin etti ama herhalde makyaj
yapmıştı. Çocuksuz kadınların böyle dalavereler çevirdiğini duymuştum da bir
adamın bunu yaptığı nerede görülmüş?''

``E sen sonra ne yaptın?''

``On bir ay daha o nemrut yerde kaldım. Üç ameliyat daha geçirdim. Dört ay sonra
artık sakallarım çıkmaya başlıyordu; hastaneden çıkmadan önce düzenli traş
olmaya başlamıştım\dots ve artık erkek olduğuma dair hiç bir şüphem
kalmamıştı.'' Acı acı gülümsedi. ``Bakışlarım hemşire yakalarından içeri
doğru kayıyordu.''

``Bana sorarsan iyi atlatmışsın.'' dedim. ``Karşımda normal bir erkek gibi
oturuyorsun, iyi para kazandığını ve bir derdin olmadığını söylüyorsun. Hem bir
kadının hayatı öyle kolay bir hayat değildir.''

Ters ters bana baktı. ``Çok biliyorsun ya!''

``Ne olmuş?''

``Hiç `acıların kadını' diye bir laf duydun mu?''

``Yıllar önce duydum sanırım. Bugün pek bir anlam ifade etmiyor.''

``Bu söz benim için söylenmiş gibiydi. O herif beni gerçekten mahvetti.
Artık bir kadın değildim\dots ve \emph{nasıl} erkek olunur bilmiyordum.''

``Alışman zaman almıştır herhalde.''

``Tahmin bile edemezsin. Ne giyeceğine veya hangi tuvalete gireceğine alışmayı
kastetmiyorum; bunları hastanede öğrenmiştim. Ama nasıl yaşayacaktım? Nasıl
ekmek parası kazanacaktım? Araba sürmeyi bile bilmiyordum. Ticaretin te'sinden
anlamazdım. Amelelik de yapamazdım. Her yerim yara bere içindeydi.''

``O heriften nefret etmemin asıl sebebi K.A.L.T.A.K'a katılmak ile ilgili
hayallerimi de mahvetmesiydi. Asıl darbeyi bunun yerine Uzay Birlikleri'ne
katılmak için başvuru yaptığımda yedim. Askeri hizmet için uygun olmadığıma
karar vermeleri için göbeğime bir bakmaları yetmişti. Tabip subay benimle
ilgilenince umutlanmıştım ama meğer o da merakındanmış. Hakkımda yazılanları
okumuş.''

``Ne edeyim, ben de adımı değiştirip New York'a geldim. Bir süre aşçılık yaptım.
Sonra bir daktilo kiralayıp stenograflık yapmaya başladım. Şaka gibiybi! Dört
ayda elime hepi topu dört mektup bir de kitap taslağı işi gelmişti. \emph{Gerçek
Hayattan Efsaneler} adında bir kitaptı ve kağıt israfından başka bir şey değildi.
Ama yazarı olacak o mankafanın kitapları satılıyordu. Aklıma bir fikir gelmişti.
Bir yığın itiraf dergisi aldım, hepsini okuyup inceledim.''

Gözlerinde alaycı bir bakış vardı. ``Böyle işte. Evlenmemiş Anne hikayemi
dinledin. Yazıp da satmadığım ve gerçek olan tek hikayemi. Artık kadın bakış
açısından neden iyi anladığımı biliyorsun. İddiayı kazandım herhalde?''

Şişeyi ona doğru ittim. Hâline üzülmüştüm ama yapmam gereken de bir iş vardı.
``Bana bak evlat, şu herifin hâlâ yakasına yapışmak istiyor musun?''

Gözleri ışıldadı. Vahşi bir parıltıydı bu.

``Dur bakalım!'' dedim. ``Onu öldürmeyeceksin herhalde?''

Acımasızca sırıttı. ``Gör bakalım.''

``Sakin ol. Düşündüğünden fazlasını biliyorum. Sana yardım edebilirim. Onun
nerede olduğunu biliyorum.''

Barın karşısından uzanıp yakama yapıştı. \emph{``Nerede o herif?''}

İstifimi bozmadım. Usulca, ``Bak yakamı bırak ufaklık, yoksa seni şuracıkta yere
sererim; polis gelince de sızıp kaldığını söyleriz.'' dedim ve sopayı gösterdim.

Ellerini çekti. ``Kusuruma bakma ama söylesene nerede?'' dedi. Yüzünü bana
çevirdi. ``Daha fazlasını nereden öğrendin?''

``Anlatacağım sakin ol. Kayıtlarını okudum: Hastane kayıtları, yetimhane
kayıtları, tıbbi kayıtlar\dots Yetimhanenizin başhemşiresi Bayan Fetherage'dı
değil mi? Sonra da yerine Bayan Gruenstein geldi doğru mu? Kızken adın `Jane'di
değil mi? Üstelik bunların hiç birini de bana söylemediğine eminsin, öyle değil
mi?''

Onu şaşırtmış ve biraz olsun korkutmuştum. ``Bunlar da ne demek oluyor? Başımı
belaya mı sokmak istiyorsun?''

``Hayır aksine iyiliğini düşünüyorum. Bu adamı senin ellerine verebilirim. Nasıl
bilirsen öyle yaparsın. Sonrasında da paçayı kurtaracağına söz veriyorum.
Yalnızca, onu öldüreceğini düşünmüyorum. Bunun için gerçekten deli olman gerek
ve sen deli değilsin. Yani demek istediğim o kadar deli değilsin.''

Kadehi kafaya dikip tezgâha koydu. ``Uzatma da söyle artık. Nerede o?''

İçkisini tazeledim; sarhoş olmuştu ama öfkesi onu ayık tutuyordu. ``Ağır ol
bakalım. Senin için bunu yapacaksam karşılığında sen de benim için bir şey
yapacaksın.''

``Of\dots Ne?''

``İşini sevmiyorsun. Yerine dolgun ücretli, harcamalarının tümünün karşılandığı,
kimseden emir almadığın, düzenli bir işe ne dersin? Yaşayacağın türlü tecrübe ve
maceralar da cabası!''

Dik dik baktı. ```Al işini başına çal!' derim. Amma attın ha, nerede o günler?''

``Peki madem şöyle diyelim: Ben senin için o adamı bulayım. Derdin neyse hallet.
Sonra sen de bahsettiğim işi dene. İddia ettiğim kadar değilse seni tutacak da
değilim hani.''

Son kadeh işini görmüştü. Sallanıyordu. ``Oh, onu neya zaman getireceksiin?''

``Anlaştık dersen \emph{hemen şimdi!}''

Elini uzattı. ``Anlaştık!'' dedi.

Yardımcı barmene iki tarafa da göz kulak olmasını söyledim. Zamanı not
ettim. Saat 23.00. Eğilip barın altındaki açık bölmeden diğer yana geçerken
müzik kutusundaki şarkı \emph{``Ben kendimin  dedesiyim!''}\footnote{%
Ç.N: Yazar, ``I'm my Own Grandpa!'' adlı, sözleri Dwight Latham ile Moe Jaffe
tarafından yazılan ve ilk defa 1947 yılında Lloyd ``Lonzo'' George (1924-1991)
ve Rollin ``Oscar'' Sullivan (1919-2012) tarafından yorumlanan şarkıdan
bahsetmektedir. \citep{WkGrandpa12} Eser, ülkemizde bir dönem Grup Vitamin
tarafından da icra edilen, ``novelty song'' adıyla anılan müzik türündedir.

Eserin sözleri, olasılığı düşük de olsa yasal olan bir dizi evlilik sonrasında
kendi üvey annesinin üvey babası olan bir adamı konu almaktadır. ``Üvey''
nitelendirmeleri kaldırıldığında ise bahsedilen eserin kahramanı kendi dedesi
olmaktadır. \citep{WkGrandpa12}

Daha detaylı bilgi için \citealp[bkz.][]{alipGrandpa13}
}
diye içeriyi inletmeye başladı. Garsona 70'li yılların ``şarkılarını'' midemin
kaldırmadığını söylemiş ve müzik kutusunu eski americana\footnote{%
Ç.N: Americana: Amerika Birleşik Devletleri'nin tarihine ve kültürüne ait olan
eserler. \citep{TrAmericana}
}
ve klasiklerle
doldurmasını tembihlemiştim ama aralarında bu şarkının da olduğunu bilmiyordum.
``Kapat şu zımbırtıyı! Müşteriye de parasını iade et!'' diye bağırdım ve ekledim.
``Ben kilere gidiyorum, birazdan geri dönerim.'' Evlenmemiş Annemi de alıp
kilere doğru yürümeye koyuldum.

Kiler, tuvaletlerin karşısındaki koridorun sonundaydı. Çelik kapısının anahtarı
yalnızca bende ve gündüz müdüründe vardı; oradan da anahtarı sadece bende olan
bir kapıdan arka bir odaya geçiliyordu. Evlenmemiş Anne ile beraber bu arka
odaya girdik.

Uykulu gözlerle odayı çevreleyen penceresiz duvarlara baktı. ``Nerde bu herif?''

``Hemen geliyor.''

Odanın içindeki tek şey olan çantayı açtım: A.B.D. Yakıt Kurumu Dönüştürücü
Alan Teçhizatı, 1992 serisi, 2. Model: sabit aksamları, pilleri tam dolu iken
yirmi üç kilogramlık ağırlığı ve bir bavul şeklinde taşınabilir tasarımıyla tam
bir ay parçası. O gün erkenden aletin ince ayarını yapmıştım. Tek yapmamız
gereken dönüştürücü alanı kısıtlayan metal ağın içine girmekti.

Ağı kaldırdım. ``Bu da neyin nesi?'' diye sordu.

``Zaman makinesi.'' diye yanıtladım ve kaldırdığım metal ağı üzerine savurdum.

``Hey!'' diye bağırıp geri adım attı. Bu işin bir tekniği vardır: Ağı öyle
atarsınız ki hedef içgüdüsel olarak geri adım attığında kendini kafesin içinde
bulur. İkiniz de kafesin içindeyken ağı bütünüyle kapatırsınız. Yoksa bazen bir
ayakkabı topuğunu, hatta bazen bir ayağı dahi bütünüyle geride bırakabilir ya da
zeminin bir parçasını yanınızda götürebilirsiniz. Güzel tarafı makineyi
kullanmak için bunun dışında bir beceriye ihtiyacınız yok. Bazı ajanlar
hedeflerini ağın içine sokmak için onlara türlü numaralar çevirir. Bense
yaşadıkları anlık, büyük şaşkınlıktan faydalanıp onları kafesin içine alır ve
makineyi çalıştırırım. Şimdi de aynen böyle yaptım.

\noindent\emph{10.30-6. Zaman Dilimi-3 Nisan 1963-Cleveland, Ohio-Apex Apt.:}

``Hey!'' diye yineledi. ``Kaldır şu şeyi üzerimden!''

``Üzgünüm,'' diyerek özür diledim ve ağı kaldırıp çantanın içine tıktıktan sonra
çantayı kapattım. ``Onu bulmak istediğini söylemiştin.''

``İyi ama sen de bunun bir zaman makinesi olduğunu söyledin!''

Ona bir pencereden dışarısını gösterdim. ``Hiç Kasım ayına benziyor mu, veya New
York şehrine?'' O, dışarıda ilkbahar havasında yeşeren otlara aval aval bakarken
ben de çantayı tekrar açıp içinden bir tomar yüz dolarlık banknot çıkardım, seri
numaraları ve imzaları 1963 senesiyle uyumlu mu diye kontrol etmeye koyuldum.
Zaman Bürosu ne kadar para harcadığınızı umursamaz (onlara bir maliyeti yoktur)
fakat gereksiz tarih hataları yapmanızdan hoşlanmaz. Böyle çok hata yaparsanız
genel bir askeri mahkeme sizi bir sene boyunca nahoş bir döneme sürgüne
gönderir, örneğin insanların zorunlu olarak çalıştırıldığı ve ihtiyaçların karne
uygulamasına tabi tutulduğu 1974 senesine. Asla böyle hatalar yapmam.
Banknotlarda bir sorun yoktu.

Dönüp bana baktı: ``Ne oldu?'' diye sordu.

``Aradığın adam burada. Git ve onu bul. Al bunlar da harcaman için.''

Parayı uzattıktan sonra ekledim. ``Aranızdaki mevzuyu hallet. Sonra gelip seni
alacağım.''

Alışık olmayan birine yüz dolarlık banknotları ilk defa gösterdiğinizde hipnoz
olmuşa döner. İnanamadı ve paraları saymaya girişti. Ben de onu koridora
çıkardım ve geri giremesin diye kapıyı içeriden kilitledim. Bir sonraki
sıçrayışım kolaydı: Yalnızca küçük bir dönem değişikliği.

\noindent\emph{17.00-6. Zaman Dilimi-10 Mart 1964-Cleveland, Ohio-Apex Apt.:}

Kapının altında kira sözleşmemin iki hafta sonra bittiğini söyleyen bir not
buldum. Bunun haricinde odada bir an öncesine göre bir değişiklik yoktu.
Dışardaysa ağaçlar yapraklarını dökmüştü ve kar yağıyordu. Güncel banknotlarımı
ayarladım. Odayı kiralarken orada bıraktığım ceket, şapka ve pardesümü
giydikten sonra vakit kaybetmeden işe koyuldum. Bir araba kiraladım ve
hastaneye gittim. Yardımcı hemşirenin kafasını ütülemeye başladım. Yirmi dakika
kadar sonra artık bıkmış olan hemşire beklediğim fırsatı bana verdi. Kimse
farketmeden bebeği alıp oradan çıktım. Beraber Apex Apartmanı'na gittik.
Sonraki adım için gereken ayarlamalar daha uğraştıcıydı çünkü bu bina 1945
yılında henüz inşa edilmemişti. Ama bunları önceden hesaplamıştım.

\noindent\emph{01.00-6. Zaman Dilimi-20 Eylül 1945-Cleveland, Skyview Moteli:}

Makine, bebek ve ben şehir dışındaki bir motele vardık. Evvelden burada
``Warren/Ohio'dan Gregory Johnson'' adıyla bir oda tutmuştum. Bu yüzden
vardığımızda odanın perdeleri kapalı, pencereleri kilitli, kapısı sürgülüydü ve
zemini de makinenin inişini kolaylaştırmak için boşaltılmıştı. Yanlış yerde
unutulan bir sandalye insanın canını fena yakabilir. Tehlike sandalyenin
kendisinden değil elbette, alanın onu geri tepmesinden ötürü.

Sorun yaşamadık. Jane mışıl mışıl uyumaya devam ediyordu. Onu dışarıya taşıdım.
Daha önceden temin ettiğim bir arabanın arka koltuğundaki alışveriş sepetine
koydum. Yetimhaneye götürdüm ve merdiven basamaklarına bıraktım. İki sokak
ilerideki servis istasyonuna (yakıt satan türden) gittim. Yetimhaneyi aradım ve
görevliler sepetteki bebeği içeri götürürlerken izlemek için arabayla geri
döndüm. Sonra sürmeye devam ettim. Arabayı motelin yakınlarında bir yerde
bıraktım. Odama girdim. 1963 yılındaki Apex Apartmanı'na sıçradım.

\noindent\emph{22.00-6. Zaman Dilimi-24 Nisan 1963-Cleveland, Ohio-Apex Apt:}

Duracağım anı iyi ayarlamıştım. Varış noktası sıfır noktasından farklıysa
hesapları zaman aralığına göre ayarlamak gerek. Eğer bir hata yapmadıysam şu an
Jane bu huzurlu ilkbahar gecesinde bahsettiği parkta düşündüğü kadar ``uslu''
bir kız olmadığının farkına varıyor olmalıydı. Bir taksi tutup şu pintilerin
oturduğu evin sokağına gittim. Şöföre köşede beklemesini söyleyip gölge bir
yerde pusuya yattım.

Bir süre sonra ikisini sokağın başından sarmaş dolaş gelirlerken gördüm.
Kapının önüne gelince de adam kıza uzun bir iyi geceler öpücüğü verdi.
Öpüşmeleri sandığımdan uzun sürdü. Kız eve girdikten sonra adam yürümeye devam
edip uzaklaşmaya başladı. Gidip koluna girdim ve onu sürüklemeye başladım.
``Benden bu kadar evlat.''  dedim. ``Seni almaya geldim.''

\emph{``Sen!''} derken nefesi kesildi. O durup soluklanırken ben lafa girdim.

``Evet, ben. Şimdi onun kim olduğunu biliyorsun ve biraz düşünürsen kendinin de
kim olduğunu anlarsın\dots ve yeterince düşünürsen bebeğin kim olduğunu da
anlarsın\dots ve \emph{benim} kim olduğumu da.''

Cevap veremedi. Derinden sarsılmıştı. Doğal tabi, kendisi tarafından
ayartılmaya dayanamayacağını yüzüne vurmuştum. Onu Apex Apartmanı'na geri
götürdüm. Zamanda yeniden sıçradık.

\noindent\emph{23.00-7. Zaman Dilimi-12 Ağustos 1985-Rocky Dağları Askeri Üssü:}

Nöbetteki çavuşu uyandırdım. Kimliğimi gösterdim. Ona, yanımda getirdiğim
dostuma yatıştırıcı bir hap verip uyutması ve sabah da birliğe kabulünü yapması
için emir verdim. Çavuş somurttu ama hangi zamanda olursanız olun rütbe
rütbedir. Dediğimi yapacaktı fakat şüphesiz aklında bir dahaki karşılaşmamızda
bu sefer kendisinin albay benimse çavuş olacağım umudu vardı. Bu türden
değişiklikler bizim birliklerde olabilir. ``Adı ne?'' diye sordu.

Adını yazdım. Kaşlarını kaldırdı. ``Nasıl? \emph{Ha?}\dots''

``Yalnızca işini yap çavuş.'' dedim. Arkadaşıma döndüm.

``Evlat, artık dertlerin sona erdi. Hayal bile edemeyeceğin bir iş buldun. Bu
işte iyi olacaksın. \emph{Biliyorum.''}

Çavuş, ``kesinlikle!'' diyerek onayladı: ``Baksana bana, 1917 doğumluyum. Hâlâ
ayaktayım, hâlâ gencim ve hayatın tadını çıkarıyorum.'' Sıçrama odasına döndüm.
Bütün ayarları daha önceden belirlenmiş sıfır değerlerine geri aldım.

\noindent\emph{23.01-5. Zaman Dilimi-7 Kasım 1970- New York, "Baba'nın Yeri":}

Birkaç dakikalık yokluğumun bahanesi olarak kilerden elimde boşa yakın bir
Drambuie\footnote{Ç.N: Bal ve viskiden yapılan bir tür İskoç likörü} şişesiyle
çıktım. Yardımcım \emph{``Ben kendimin dedesiyim!''} şarkısını çalan müşteri ile
tartışıyordu. ``Bırak çalsın, sonra da şu müzik kutusunun fişini çek.'' dedim.
Çok yorulmuştum.

Zor iş olduğu doğru ama birinin yapması gerek. Özellikle son yıllarda, 1972
Hatası'dan beri birliğe yeni insanlar katmak epey zorlaştı. Dolgun maaşlı ve
(tehlikeli olsa da) ilgi çekici bir işte insanların iyiliği için çalışacak
birilerini ararken bulundukları yerde kafası karman çorman olmuş insanlardan
daha iyi bir kaynak düşünebiliyor musunuz? 1963 yılındaki Başlamadan Biten
Savaş'ın adının nereden geldiğini herkes bilir. Üzerinde New York şehir posta
kodunun yazılı olduğu bomba infilak etmemişti. Bununla birlikte planlanan her
şey de ters gitmişti. Bu tersliklerin hepsini biz tertiplemiştik.

Ancak `72 Hatası öyle değildi. Bu bizim suçumuz değildi ve geri alabileceğimiz
bir şey de değil. Ortada çözülmesi gereken bir paradoks yok. Bir şey ya vardır
ya da yoktur, şimdi ve sonsuza dek, amin. Bir daha böyle bir şey asla
olmayacak: ``1992'' yılında verilen emrin diğer bütün yıllarda verilen emirlere
göre önceliği var.

Barı normalden beş dakika önce kapattım. Gündüz müdürüne barın bana ait olan
hisselerini ona satmayı kabul ettiğimi belirten bir not yazdım ve uzun bir
tatile çıkacağım için konuyu avukatım ile halletmesini rica ettim. Notu yazar
kasanın içine koydum. Büro, adamın yapacağı ödemeyi alır mı almaz mı bilemem
ama giderken arkamda bir sorun bırakmadan ayrılmamı isteyecekleri kesin.
Yeniden kilerin arkasındaki odaya gittim ve 1993 senesine sıçradım.

\noindent\emph{22.00-7. Zaman Dilimi-12 Ocak 1993-Rocky Dağları, Zaman İşleri Askeri Birliği:}

Nöbetçi subaya göründükten sonra odama geçtim. Bir hafta boyunca dinlenme
niyetindeydim. İddiasına girdiğimiz şişeyi (sonuçta kazanmıştım) yanımda
getirmiştim. Raporumu yazmaya başlamadan önce bir kadeh içtim. Tadı berbattı.
Nasıl zamanında Don Murdar'ı seviyordum şaştım. Ama hiç yoktan iyi olduğu da
kesindi. Tamamen ayık olduğum zaman her şeyi kafaya takmaya başlarım.  Şişenin
dibini görecek kadar da içecek değilim ama. Herkesin olduğu gibi benim de
yapacak işlerim vardı.

Raporumu tamamladım. Birliğe alınması talebinde bulunduğum kırk kişi de
Psikoloji Bürosu'nun onayından geçmişti. Bunlara benim başvurum da dahildi.
Zaten onun kabul edileceğinden emindim. Sonuçta buradaydım, öyle değil mi? Daha
sonra müdahele bölümüne geçiş yapmak için bir dilekçe doldurdum. Personel alımı
işiyle uğraşmaktan bıkmıştım. Raporu ve dilekçeyi teslim ettikten sonra dönüp
yatağıma uzandım.

Yatağımın yanında asıl duran ``Zaman Yönetmelikleri'' belgesi gözüme çarptı:

\begin{verse}
\itshape
  Sakın Dün Yapma,\\
  \hspace{1em}Yarın Yapılması Gerekenleri.\\
  Başarılı Olursan Sonunda,\\
  \hspace{1em}Sakın Yeniden Deneme.\\
  Zamana Atılan Bir Dikiş,\\
  \hspace{1em}Bütün İnsanlığı Kurtarır.\\
  Her Paradoksu Savacak,\\
  \hspace{1em}Bir Paratoner Bulunur.\\
  Vakit Hep Erkendir,\\
  \hspace{1em}Düşünüp Hareket Edersen.\\
  Geçmişte Yaşayanlar da İnsandır,\\
  Yer Gök Şahittir.\\
\end{verse}

Birliğe ilk kabul edildiğim zamanki kadar beni heyecanlandırmıyordu artık bu
sözler; zaman sıçramalarıyla geçirdiğim bu otuz yıllık göreli zaman beni yordu.
Üzerimi çıkarıp yattıktan sonra göbeğime baktım. Artık o kadar tüylenmişti ki
sezaryenden kalan yara izlerini dikkatli bakmadan farkedemiyordum.  Sonra elimdeki
yüzüğe baktım. Sonu olmayan, sonsuza dek kendi kuyruğunu yiyen koca yılan\dots
Ben nereden geldim biliyorum da ya siz zombiler, ya sizler nereden geldiniz?

Başıma bir ağrı girmeye başladı ama ağrı kesici içmeyecektim. Bir daha
kullanmayacağım bir ilaç varsa o da ağrı kesicidir. Bir keresinde içmiştim ve
hepiniz ortadan kaybolmuştunuz.

Yorganın altına girdim. Bir ıslık çalıp ışıkları söndürdüm.

Aslında \emph{hiçbiriniz} yoksunuz. Benden - Jane'den - başka kimse yok.\\
Yapayalnızım bu karanlıkta.

Sizleri öyle çok özledim ki!
