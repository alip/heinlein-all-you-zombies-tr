% vim: set filetype=tex fileencoding=utf8 et sw=2 ts=2 sts=2 tw=80 :
% © 2014 Ali Polatel <polatel@gmail.com>
% Creative Commons Attribution-NonCommercial-ShareAlike 3.0 Unported Lisansı ile yayınlanmıştır.

\chapter{Çevirinin Değerlendirilmesi}

Çeviri öncesinde yapılan kabullere göre, çeviri süreci boyunca düz ve
yananlamsal eşdeğerlilik ile beraber dil-kullanımsal ve biçimsel eşdeğerliliğin
de (kavramlar için \citealp[bkz.][]{GokturkEsdeger}) sağlanmasına çabalanmıştır.
Kaynak metne bağlılık ise hikayenin değişken doğası sebebiyle her noktada aynı
ölçüde değildir.

Çeviriyi yaparken yukarıda bahsi geçen eşdeğerlilikleri sağlamak adına belli
noktalarda ifadeler erek dile uyarlanmış, diğer noktalarda ise buna gerek
görülmemiştir. Bu iki duruma birer örnek aşağıda verilmiştir.

\begin{enumerate}
  \item
    Kaynak metinde ana kahramanın katılmayı planladığı \emph{W.E.N.C.H.} adlı kurum,
    farklı dönemlerde \emph{A.N.G.E.L.} ve \emph{W.H.O.R.E.} şeklinde de anılmıştır.
    Son kısaltma metinde direkt olarak belirtilmemiş fakat ilk iki kısaltmanın
    açılımlarına değinildikten sonra kurumun \emph{üçüncü bir ismi} de olduğu ifade
    edilerek açılımı verilmiştir: \emph{Women's Hospitality Order Refortifying \&
    Encouraging Spacemen}. İngiliz dilinde \emph{backronym} olarak adlandırılan bu
    kelime oyunu, ``var olan bir kısaltma veya kelimenin gerçek kullanımına
    dayanmayan şekilde açılmasıdır'' \citep{OxfordBackronym}.

    Bu tür kısaltma açmalarına İngiliz dilinden verilebilecek uygun bir örnek
    \emph{Erasmus Programı}'ndaki \emph{Erasmus} kelimesinin kullanımıdır. Özel bir
    isim olan Erasmus kelimesi, adını kullanan programın amacını da belirtmesi için
    ``European Community Action Scheme for the Mobility of University Students''
    şeklinde açılmıştır \citep{WkErasmus}. Dilimizde \emph{kin} kelimesinin
    ``kişinin içindeki nefret'' \citep{SesliBackronym} olarak açılması da benzer bir
    durumdur.

    Kısaltmaların dipnot ile açıklanması çevirinin yeterliliğiyle ilgili iki
    sorun oluşturmaktadır:

    \begin{enumerate}
      \item
        Hikayede kısaltmalar birden fazla yerde geçmektedir. Dolayısıyla dipnot ile
        açıklama erek dildeki okur için hikayenin akıcılığını zedeler.
      \item
        Yazar üçüncü kısaltmanın yalnızca açılımını vermiş, baş harflerinden bir
        kelime oluşturup oluşturmamayı okura bırakmıştır. Dolayısıyla verilecek
        dipnot yazarın bu \emph{üstü kapalı} biçeminin erek dil okuruna
        yansıtılmasını engeller.
    \end{enumerate}

    Bu sebeplerden ötürü kısaltmaları uyarlama yoluna gidilmiştir. Kısaltmalardan
    ikisinin kaynak dilin argosuna ait oluşu da göz önünde bulundurulmuştur.
  \item
    Kaynak metinde geçen \emph{servis station} ifadesi üzerine birden fazla
    anlam yüklenen bir diğer kullanımdır. Yazar, hikayenin iki farklı dönem
    (Churchil dönemi ve anlatıcının şimdiki zamanı) için bu ifadeye birbirinden
    bütünüyle farklı iki anlam yüklemektedir. Ancak bu iki anlamın da
    Güttinger'in ortaya koyduğu şekilde ``kaynak ve erek dillerde aynı etkiyi
    yaratması'' \citep{GokturkEsdeger} için ``kelimesi kelimesine çeviri''
    \citep{NewmarkTextbook} ile erek dile aktarımı eşdeğerliliği sağlayabilir:
    \emph{(akaryakıt) servis istasyonu} ifadesi erek dilde kullanılmaktadır.
    Kelimeye yüklenen diğer anlamın da, kaynak dildeki \emph{escort
    agency/service} ifadesinin erek dile \emph{eskort servis} olarak taşınmış
    olduğu dikkate alındığında, en azından benzer bir etki yaratacağı açıktır.
\end{enumerate}

Bilim kurguya özgü olan ve/veya yazar tarafından türetilmiş ifadeler anlamları
koruyacak şekilde taşınmaya çalışılmıştır. Bunun mümkün olmadığı bir durumda -
hikayede \emph{Old Underwear} adıyla anılan içki örneğinde - erek dilde benzer
bir anlatımı yakalamak üzere yeni bir ifade türetilmiştir: \emph{Don Murdar}.
