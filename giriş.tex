% vim: set filetype=tex fileencoding=utf8 et sw=2 ts=2 sts=2 tw=80 :
% © 2014 Ali Polatel <polatel@gmail.com>
% Creative Commons Attribution-NonCommercial-ShareAlike 3.0 Unported Lisansı ile yayınlanmıştır.

\chapter{Bilim Kurgu ve Bilim Kurgu Çevirisi Üzerine}

Bilim kurgu, temelini bilimsel ve teknolojik unsurlardan alsa da yarattığı
kurgusal evreninde fiziki ve doğal evrenin sınırlarının yerini hayal gücünün
sınırları alır. Bu bağlamda tanımlanması güçtür ve geniş, belirsiz sınırlarının
içinde birçok alt tür barındırmaktadır.

Döneminin en başarılı bilim kurgu yazarlarından biri olarak kabul edilen ve
adından sıklıkla ``bilim kurgu yazarlarının duayeni'' \citep{SFH09}
olarak söz ettiren \textsc{Robert A. Heinlein}'ın (1907--1988) yaptığı bilim
kurgu tanımı, türün genel hatlarını betimlemektedir: ``Neredeyse bütün bilim
kurguyu kapsayan kullanışlı, kısa bir tanım şöyle olabilir: Temelini gerçek
dünya, geçmiş ve şimdiki zaman hakkında yeterli bilgi ve doğa ile bilimsel
yöntemlerin kapsamlı bir anlayışı üzerine sağlam olarak oturtan, gelecekteki
olası olaylar üzerine yapılan gerçekçi kurgulamalar.'' \citep{SFN59}
Dilimizdeki \emph{bilim kurgu} ifadesinin \emph{isim
babası} olan yazar ve gazeteci \textsc{Orhan Duru} (1933--2009) da bilim kurguyu
benzer şekilde tanımlamıştır: ``Bilim-Kurgu'yu tanımlamak çok zor. Kişinin ve
yazarın görüşüne göre değişiyor.  Bu tanım, gerçeklerle, bilimsel verilerle bir
ölçüde sınırlı bir düşçülük diyebiliriz bilim-kurguya.'' \citep{XBil76}

Bilim ve bilimin \emph{omuzlarında yükselen} bilim kurgu arasındaki etkileşim,
yazın türünün doğuşundan bugüne kadar süregelmiştir. \textsc{Schwartz}, \emph{Bilim
Kurgu: İki Kültür Arasındaki Köprü} yazısında bu ilişkiyi örneklendirerek
açıklar: ``Bilim ve kurgu arasında iki yönlü sürekli bir etkileşim vardır,
örneğin \emph{Ray Bradbury: Story of a Writer (Sterling)} filminde yazarın yeni
kısa hikayesi için gerekli bilgiyi elde etmek üzere bir telefon bilgisayar
merkezine gittiğini görürüz. Düzyazı geleneğimizde ilk defa baskın hale gelen bu
emsalsiz ilişkiden ötürü bilim kurgu çağımızı en iyi yansıtan edebi tür olarak
görülebilir.'' \citep{schwartz71}

Günümüzden geriye bakıldığında bilim kurgunun bilim için bir esin kaynağı, bir
tetikleyici güç olduğu görüşü gitgide daha fazla benimsenmektedir. Teorik
fizikçi \textsc{Michio Kaku} bu durumdan coşkuyla söz eder: ``Sicim teorisi
gibi gelişmiş teorilerin ortaya çıkmasıyla birlikte zaman yolculuğu ve paralel
evrenler gibi yalnızca bilim kurguya ait olan kavramlar dahi artık fizikçiler
tarafından yeniden değerlendiriliyor. Yüz elli sene önce zamanın bilim insanları
tarafından `imkansız' olarak nitelenen ve şimdi günlük hayatımızın birer
parçası olan teknolojik gelişimleri bir düşünün. Jules Verne'in 1865 yılında
yazdığı \emph{Yirminci Yüzyılda Paris} adlı kitap bir yüzyıldan fazla bir süre
sonra torununun oğlu tarafından tesadüfen bulunana ve 1994 yılında ilk kez
basılana dek saklı kalmıştı. Verne, kitapta Paris'in 1960 yılında nasıl
görüneceğini tasvir ediyordu. Romanı on dokuzuncu yüzyılda açıkça imkansız
olduğu ifade edilen teknolojilerle doluydu: Faks makineleri, dünya çapındaki bir
iletişim ağı, cam gökdelenler, doğalgazlı arabalar, yüksek hızlı trenler.''
\citep{PhyIMP09}

Bilim kurgu, yaratıcı ve üretken yapısıyla yazıldığı dillere de katkı yapmakta,
yeni kelime ve deyişlerin türetilmesinin öncüsü olmaktadır. Örnek olarak
İngilizce diline \textsc{Robert A. Heinlein} tarafından kazandırılan
\emph{``grok''} kelimesi gösterilebilir. \textsc{Stranger in a Strange Land}
adlı kitabın birçok yerinde geçen kelimenin yine aynı kitap içinde tanımı da
verilmiştir: ```Grok' o denli derinlemesine kavrayıştır ki gözlemci, gözlenenin
bir parçası olur: Birleşir, uyum sağlar, birbirleriyle evlenirler ve kimlikleri
birliklerinin içinde kaybolur. Din, felsefe ve bilim derken aklımıza gelenlerin
neredeyse hepsidir -- ve bize ifade ettiği anlamın, renklerin kör bir adam için
ifade ettiği anlamdan bir farkı yoktur.'' \citep{SSL61} Kelime, başta okur
kitlesi sonrasında da toplum tarafından kabul görmüş, kullanılmaya başlanmış ve
sözlüklerde yerini almıştır. Kelimenin \emph{Oxford Dictionary of English}
sözlüğündeki tanımı: ``(bir şeyi) sezgiyle veya empatiyle anlamak''tır
\citep{Oxford05}.

Çeviribilim çerçevesinde bilim kurgu incelenecek olursa hali hazırda
tanımlanması zor olan kaynak metnin erek dil ve kültür çerçevesinde işlev ve
amacının belirlenmesi gerekecektir. Bu açıdan ortaya çıkacak sorunlar genel
olarak bakıldığında teknik çevirilerde \citep{Aksoy1998} ve edebi çevirilerde
karşılaşılan sorunların \citep{BengiÖner1999} bir harmanı şeklindedir. Çevirmen,
bir yandan kurgunun temelinde var olan teknik alt yapıyı diğer yandan da eserin
edebi havasını erek dile taşımak durumundadır. Kaynak metnin derinlemesine
analiz edilmesi, çeviri metninin erek dildeki işlevi ve amacının belirlenmesiyle
birlikte çeviri için bir plan ve yöntem seçilmesi asıl çeviri işlemine
başlamadan önce yapılmalıdır.

\textsc{Robert A. Heinlein}'ın ``All You Zombies--'' adlı kısa hikayesi, günlük
bir dilde yazılmıştır ve özellikle giriş bölümünde sık olarak argo kullanımına
başvurulmuştur. Yazar bu anlatım şekliyle okuyucusunu içinde bulunduğu dünyadan
yavaşça hikayenin içine çeker. Ardından da \emph{zaman yolculuğunun} çelişkili
doğası adım adım okuru sarar. Teknik alt yapı olarak neredeyse hiçbir art-alan
bilgisi öngörmez. Yazar okurun ihtiyacı olan bütün bilgiyi - kimi zaman açık
bir dille, kimi zaman da üstü kapalı olarak \emph{satır aralarında} -
vermektedir. \textsc{H.G. Wells}'in 1895 yılında yazdığı bu alt türün öncüsü
kabul edilen ``Zaman Makinesi'' romanının aksine ``All You Zombies--''
hikayesindeki anlatım şekli yazı boyunca aynı kalmaz. Yazarın yaptığı hızlı
değişiklikler akıcılığı körüklemekte, \emph{zaman yolculuğunun} fiziksel
imkansızlıkları ve felsefik çapraşıklıklarını \emph{sıradan} bir dünyada yaşayan
okurun yüzüne vurmaktadır.

Kaynak metin ile ilgili yapılan bu analiz ve ilgili kabuller ışığında çeviri
için yöntemler belirlenmiştir. Çevirmenin öncelikli amacı bahsi geçen bu kafa
karıştırıcı mizahı erek dile taşımaktır. Yazarın \emph{üstü kapalı} kalmasını
yeğlediği detaylar için erek metinde herhangi bir ek açıklama yapılmayacaktır.
